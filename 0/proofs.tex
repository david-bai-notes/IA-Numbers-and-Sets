\section{Introduction: Proofs and why we want them}
Vaguely speaking, a proof is a kind of logical argument (or a series of it) which establishes a conclusion.
This is not the most rigorous way to define it, but it would be enough for now.
The big question, however, is why we need them.
Why can't we just assume something that looks nice to be correct?
Why can't we declare something to be right if it holds merely for the scope that we can reach?
Of course, there is an answer.
If not, mathematics would not have existed.
\subsection{Why we need proofs}
There are a few reasons why proofs are important and essential.
The obvious one is that we need to be sure about the truthfulness of something.
There are tonnes of example in mathematcis where a theorem holds for a shockingly large selection of numbers but not for all.
We could, of course, give some silly examples like the claim ``all positive integers are less than $N$'' where you can substitute $N$ for some very large integer, but there is a more interesting example:
\begin{example}[Polya's Conjecture]
    For any positive integer $N>2$, at least half of the positive integers less than or equal to $N$ has an odd number of prime factors (counting multiplicity).
\end{example}
\begin{proof}[Disproof]
    The smallest counterexample occurs at $N=906150257$.
\end{proof}
We do have computers now that can test some of our conjectures up to some very large numbers, but for mathematicians, most of those computational powers are useless,
\footnote{There are some proofs that harness the power of computers, like the four-color theorem. But still -- it is the proof we need, not any sort of verification.}
since they can do nothing about how the general picture is like.
\subsection{Why we want proofs}
We can need something that we don't want, like nuclear weapons.
However, this is not the case for proofs.
We need them, yes.
But we still want them, desperately.
Some say that the study of language is proofreading, then the study of mathematics would be proof-reading.
What we really want is not only the however elegant results that we, or someone else, have proven.
We should be more attracted to the (clever) idea behind.
Why? Because if we come through another statement of similar kind, it would be in our advantage to solve them using the same technique.
After all, mathematics is a creative art (of problem-solving).
Quoting theorems won't get you anywhere in maths, what you really need is to empower yourself with the tricks in the proof.
\subsection{Correct and incorrect proofs}
First let us see how a correct proof is done:
\begin{claim}
    For any integer $n$, $n^3-n$ is a multiple of $3$.
\end{claim}
Note that when you state some variable, you need to state as well where its home is at.
The statement would be false if $n$ is not restricted to integers.
Now here comes the proof.
\begin{proof}
    $n^3-n=(n-1)n(n+1)$. Since one of any three consecutive integers is a multiple of $3$, one of $n-1,n,n+1$ is. So $n^3-n$ is a multiple of $3$.
\end{proof}
A proof almost always comes with a smart idea.
The ``smart'' here doesn't necessarily mean that it is really smart, but a very important point that constitutes the basic of (part of) the proof.
In this case, the smart idea is to try and factorize the expression to something that we can easily handle, which makes the problem quite obivous.\\
Now here come a non-proof, or incorrect proof:
\begin{claim}
    Let $n$ be a integer. If $n^2$ is even, so is $n$.
\end{claim}
\begin{proof}[Non-proof]
    If $n$ is even, then $n=2k$ for some integer $k$, so $n^2=4k^2=2(2k^2)$. Therefore $n^2$ is even.
\end{proof}
Is it factually wrong? No.
The logic in the proof itself is impeccable.
So both the statement and the proof logic is correct, what is wrong?
Quite obviously, it has proved the wrong thing.
We want something like $A\implies B$, but what has been proven is $B\implies A$.
So even both the statement and the proof logic are alright, it might still not be a proof.
(Needless to say, if either of these two things is not, it would not be a proof either.)
So back on the track.
\begin{proof}
    If $n^2$ even but $n$ is not, then $n=2k+1$ for some integer $k$, so $n^2=4k^2+4k+1=2(2k^2+2k)+1$ which is odd. This is a contradiction.
\end{proof}
This is a new way of proof. We do not take $A\implies B$ directly, but we use $\lnot B\implies\lnot A$.
The equivalence of these two is called \textit{reducio ad absurdum}, and the trick to use this in the proof is called a \textit{proof by contradiction}.
\subsection{``iff''}
We want to prove the following claim:
\begin{claim}
    The solutions to $x^2-5x+6$ are $x=2$ and $x=3$.
\end{claim}
Note that this claim is actually two claims: that $2,3$ are indeed solutions to the equation, and that the equation does not have any other solutions.
Therefore, to rephrase the question, we want to prove that $x^2-5x+6=0$ \textit{if and only if} $x=2$ or $x=3$.
Here, \textit{if and only if} means, well, what it literally means, that the two statements are equivalent to each other.
They are interchangable and any one of them implies the other.
So our proof must consist of both the forward and backward implication.
Otherwise, it would be incomplete.
One way of doing it is to seperately prove both sides of the claim, but we can also use a chain of "iff"s:
\begin{proof}
    $x^2-5x+6=0\iff (x-2)(x-3)=0\iff (x=2\lor x=3)$
\end{proof}
However, if such a proof is to be used, one must make sure that the adjascent statements are indeed equivalent to each other.
\subsection{Implicit use of assumption}
The final proof error that we state here would be that sometimes we use an inappropriate assumption in the proof which make it invalid.
Consider the following claim:
\begin{claim}
    $1$ is the smallest positive real number.
\end{claim}
\begin{proof}[Nonsense]
    Let $r$ be the smallest positive real number.
    If $r<1$, then $r^2<r$, contradiction.
    If $r>1$, then $\sqrt r<r$, contradiction.
    Therefore $r=1$
\end{proof}
Why is this nonsense a nonsense?
Because the proof (and the claim) both used a wrong assumption that there exists a smallest positive real number.
But there isn't.
Actually, the above proof and the fact that $1/2<1$ provides a proof of this fact.\\
So in proofs, we cannot assume, and need to avoid assuming implicitly, anything that might not be correct.
If you get to assume something, prove it first.\\
Now that we are done with proofs, here comes the genuine stuff.