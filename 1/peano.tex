\section{The Peano Axioms}
We know what the natural numbers are, or do we?
In school, we were often told a rather vague idea of the notion of natural numbers.
We know that they are $1,2,3,\ldots$, but what actually are they?
As every mathematical discipline, we do not know what is it until we define it using axioms.
In the case of natural numbers, this is done by the Peano axioms:
\begin{definition}
    Natural numbers is a triple $(\mathbb N, 1, +1)$, where $\mathbb N$ is a set, $1\in\mathbb N$ its element, and $+1:\mathbb N\to\mathbb N$ an operation on $\mathbb N$.
    It satisfies the following axioms:\\
    1. For any $n\in\mathbb N$, $n+1\neq 1$.\\
    2. For $n,m\in\mathbb N$, if $n\neq m$, then $n+1\neq m+1$.\\
    3. Let $P$ be a proposition on $\mathbb N$. If $P(1)$ is true and $P(n)\implies P(n+1)$ for all $n\in\mathbb N$, then $P(n)$ is true for all $n\in\mathbb N$.
\end{definition}
\begin{definition}[Addition]
    We define the operation $+k$ inductively by $n+(k+1)=(n+k)+1$.
\end{definition}
\begin{proposition}
    The operation $+k$ is defined for all natural number $k$.
\end{proposition}
\begin{proof}
    Induction.
\end{proof}
Similarly, using induction, we can define multiplication, exponentiation and order ($a<b\iff\exists k\in\mathbb N, a+k=b$) in the obvious way.
\begin{proposition}
    1. $(a+b)+c=a+(b+c)$.\\
    2. $a+b=b+a$.\\
    3. $(ab)c=a(bc)$.\\
    4. $ab=ba$.\\
    5. $a(b+c)=ab+ac$.\\
    6. $a<b\land b<c\implies a<c$.\\
    7. $\lnot(a<a)$.
\end{proposition}
\begin{proof}
    Induction. Induction. Induction. Induction. Induction. Induction. Induction.
\end{proof}
There is a more useful form of induction.
Induction says that if we have some proposition $P$ such that $P(1)$ is true and $P(n)\implies P(n+1)$, then $P(n)$ is true.
But in fact, we can have a ``stronger'' induction hypothesis.
\begin{theorem}[Strong induction]
    Suppose that $P$ is a proposition on $\mathbb N$.
    If $P(1)$ is true and for any $n\in\mathbb N$,
    $$\forall k\le n, P(k)\implies P(n+1)$$
    Then $P(n)$ is true for any $n\in\mathbb N$.
\end{theorem}
\begin{proof}
    Apply our usual induction on the proposition $Q(n)$ meaning
    $\forall k\le n, P(k)$
\end{proof}
Technically, we do not need to check the base case.
But it is often safer to check it.
If we want to use strong induction on $P$, we can prove $P(n)$ assuming the case for smaller numbers, as induction says if it would help to assume $P(m)$ for some $m<n$, feel free to do so.
There are a few equivalent forms of strong induction which can be pretty useful.
\begin{corollary}
    If some $n$ has $P(n)$ false, then there exists an $n$ with $P(n)$ false but $P(m)$ true for every $m<n$.
\end{corollary}
That is, if there is a counterexample, then there is a minimal counterexample.
\begin{corollary}[Well-ordering Principle]
    If some $n$ has $P(n)$ true, then there exists a minimal $n$ with $P(n)$ true.
\end{corollary}