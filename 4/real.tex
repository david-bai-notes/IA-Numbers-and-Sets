\section{The Reals}
\subsection{The Need of the Reals}
We already have $\mathbb N, \mathbb Z, \mathbb Q$, then why are we introducing the reals?
Why don't we stop at $\mathbb Q$?
\begin{proposition}
    There is no rational $x$ with $x^2=2$.
\end{proposition}
we can assume $x$ is positive since $(-x)^2=x^2$.
\begin{proof}
    Suppose we have some rational $x$ with $x^2=2$, then suppose $x=a/b$ for some positive integers $a,b$.
    So $2b^2=a^2$, but the power of $2$ in the prime factorization of $a^2$ is even but that in $2b^2$ is odd.
    This is a contradiction.
\end{proof}
\begin{remark}
    Same argument shows that if $k\in\mathbb N$ is $x^2$ for some $x\in\mathbb N$, then $k$ is a perfect square.
\end{remark}
\begin{proof}[Alternative proof]
    Again suppose that $x=a/b$ where $a,b\in\mathbb N$ has $x^2=2$.
    So every $cx+d$ where $c,d\in\mathbb Z$ is of the form $e/b$ for some $e\in\mathbb Z$.
    So $cx+d>0\implies cx+d\ge 1/d$, but then $0<x-1<1$, so $0<(x-1)^n<1/b$ for $n$ large, which is a contradiction since all $(x-1)^n$ is of the form $cx+d$ for some integer $c,d$ using $x^2=2$.
\end{proof}
How, in $\mathbb Q$, could we say $\mathbb Q$ has a gap?
Consider the set of rationals whose squares are less than 2.
Now all of these rationals are less than $2$, so $2$ is an upper bound.
$1.5,1.42,1.415,\ldots$ are all upper bounds of this set, but there is no least upper bound.
So that establishes the notion of there being a ``gap'', which is the thing we try to eliminate in our dream model of $\mathbb R$.
\begin{definition}
    The real numbers $\mathbb R$ is an ordered field with the least upper bound property.\\
    Put it into axiom, the reals consists of a set $\mathbb R$, the binary operations $+,\times$, elements $1\neq 0$, and an order $<$ where\\
    1. $(\mathbb R,+,0)$ is an abelian group.\\
    2. $(\mathbb R,\times,1)$ is an abelian group.\\
    3. $\times$ is distributive over $+$, so $\forall a,b,c\in\mathbb R,a\times(b+c)=a\times b+a\times c$.\\
    4. For any $a,b\in\mathbb R$, exactly one of $a<b,a=b,a>b$ is true.
    Also $a<b\land b<c\implies a<c$.\\
    5. $\forall a,b,c\in\mathbb R$, $a<b\implies a+c<b+c$, $c>0\implies(a<b\implies ac<bc)$.\\
    6. (Least-upper-bound property) For any nonempty $S\subset\mathbb R$ such that it is bounded from above (i.e. $\exists M\in\mathbb R, \forall x\in S,x<M$), there is a least upper bound of $S$.
\end{definition}
\begin{remark}
    1. We can already conclude $0<1$ from axioms 1 to 5.
    Indeed, if not, then $1<0$, so $0<-1$, so $0=0(-1)<(-1)(-1)=1$, contradiction.\\
    2. We can embed $\mathbb Q$ into $\mathbb R$, but the Least-upper-bound property is false in $\mathbb Q$.\\
    3. We do need the ``nonempty'' and ``bounded above'' conditions.
    We need the former to ensure that there is indeed some element in $S$ to talk about order, and we need the latter to ensure that there is at least one upper bound.\\
    4. We could constuct $\mathbb R$ from $\mathbb Q$ and we can check that axioms 1 to 6 hold.
\end{remark}
We write $\sup S$ to denote the least upper bound of $S$.
\begin{example}
    1. $S=\{x\in\mathbb R:0\le x\le 1\}$ be the closed interval $[0,1]$.
    $2$ is an upper buond for $S$ since $\forall x\in S, x\le 2$, but $3/4$ is not since $1\in S$ but $1>3/4$.
    The least upper bound is $1$, since $1$ is an upper bound for $S$ and for every upper bound $s$ of $S$ must be at least $1$ since $1\in S$.\\
    2. $S=\{x\in\mathbb R:0<x<1\}$ be the open interval $(0,1)$.
    Again, $2$ is an upper bound but $3/4$ is not since $S\ni 5/6>3/4$.
    The least upper bound is $1$ since $1$ is an upper bound and for any upper bound $s$, if $s<1$ (note that $s<0$), we must have $S\ni (1+s)/2>s$ which is a contradiction.\\
    3. $S=\{1-1/n:n\in\mathbb N\}$, then $\sup S=1$, since $1$ is clearly an upper bound and if there is an upper bound $s\in S$ such that $0<s<1$, then we take a natural number $n>1/(1-s)$, then $1-1/n>1-(1-s)=s$.
\end{example}
Note that in example 3, we have assumed the following proposition which we shall prove now.
\begin{proposition}[Axiom of Archimedes]
    $\forall x\in\mathbb R,\exists n\in\mathbb N,n>x$.
\end{proposition}
\begin{proof}
    Suppose not, the $\mathbb N$ is bounded above, so let $c=\sup\mathbb N$, then $c-1$ is not an upper bound of $\mathbb N$, so $\exists n\in\mathbb N, n>c-1\implies\mathbb N\ni n+1>c$, contradiction.
\end{proof}
\begin{corollary}
    Let $t\in\mathbb R_{>0}$, then there is some $n\in\mathbb N$ such that $1/n<t$.
\end{corollary}
\begin{proof}
    Take $n>1/t$.
\end{proof}
So the reals do not contain infinity or infinitesimal.
\begin{remark}
    1. $\sup X$ might not be in $X$.
    E.g. $\sup(0,1)=1\notin(0,1)$\\
    2. The least-upper-bound property gives as well the existence of greatest lower bound for bounded-below subsets of $\mathbb R$.
    Indeed, suppose $S\neq\varnothing$ is bounded below, then $-S=\{-s:s\in\mathbb S\}$ is bounded above, so there is some $x=\sup-S$.
    Immediately $-x$ would be the greatest lower bound of $S$.
    We denote it by $\inf S$.
\end{remark}
\begin{theorem}
    There is $x\in\mathbb R$ such that $x^2=2$.
\end{theorem}
\begin{proof}
    Consider the set $S=\{r\in\mathbb R:r^2<2\}$ which is nonempty ($1\in S$) and bounded above ($\forall s\in S,s<2$).
    So there is some $c=\sup S$, also $1\le c<2$.\\
    Consider $c^2$, if $c^2<2$, then for any $0<t<1$, we have $(c+t)^2=c^2+2ct+t^2\le c^2+5t$, so we choose any $0<t<(2-c^2)/5$, then $(c+t)^2<2$, so $c+t\in S$ but $c+t>c$, contradicting the assumption that $c$ is an upper bound.\\
    If $c^2>2$, then for any $0<t<1$, we have $(c-t)^2=c^2-2ct+t^2\ge c^2-4t$, so we can choose $t$ such that $0<t<(c^2-2)/4$, therefore $(c-t)^2>2$, thus $c-t$ is an upper bound of $S$ as well, contradicting the fact that $c$ is the least upper bound.\\
    Therefore $c^2=2$.
\end{proof}
Similarly, for any $x>0,n\in\mathbb N$, $\sqrt[n]{x}$ exists.
\begin{definition}
    A real $x$ that is not rational is called irrational.
\end{definition}
\begin{example}
    $\sqrt 2,\sqrt 3,\sqrt 5,3+5\sqrt 2$ are irrational.
\end{example}
\begin{proposition}
    For any $a,b\in\mathbb R,a<b,\exists q\in\mathbb Q, a<q<b$.
\end{proposition}
\begin{proof}
    Suppose $a<b$ and WLOG $a,b\ge 0$.
    Choose $n\in\mathbb N$ with $1/n<|b-a|=b-a$.
    So we can find $k\in\mathbb N$, $k/n\le a$ and $(k+1)/n>a$.
    Now if $(k+1)/n\ge b$, then $1/n<|b-a|\le (k+1)/n-k/n=1/n$, contradiction.
    So $a<(k+1)/n<b$.
\end{proof}
\begin{corollary}
    For any $a,b\in\mathbb R,a<b,\exists i\in\mathbb R\setminus\mathbb Q, a<i<b$.
\end{corollary}
\begin{proof}
    $\exists q\in\mathbb Q, a/\sqrt 2<q<b/\sqrt 2$, so $i=q\sqrt 2$ works.
\end{proof}
\subsection{Sequences and Their Limits}
What does $1+1/2+1/4+1/8+\ldots=2$ mean?
Why does $0.33333\ldots=1/3$?
When we come to think about it, we mean that, for example in the first case, $1,1+1/2,1+1/2+1/4,\ldots$ ``$\to$'' $2$.
But what do we mean by that?
We do not mean that the sequence will be eventually $x$, but can be ``arbitrarily close'' to that.
\begin{definition}
    The absolute value function $|x|$ is defined by
    $$\forall x\in\mathbb R,|x|=
    \begin{cases}
        x\text{, if $x\ge 0$}\\
        -x\text{, otherwise}
    \end{cases}$$
\end{definition}
Regarding absolute value, we also have the triangle inequality
\begin{proposition}
    $|x-y|\le|x-z|+|z-y|$
\end{proposition}
\begin{proof}
    Trivial.
\end{proof}
\begin{definition}
    For a sequence $(x_n)$ of reals and $x\in\mathbb R$, we say $x_n\to x$ (as $n\to\infty$) or
    $$\lim_{n\to\infty}x_n=x$$
    if $\forall\epsilon>0,\exists N\in\mathbb N,\forall n>N,|x_n-x|<\epsilon$.
\end{definition}
So it means $x_n$ will ``eventually'' go $\epsilon$-close to $x$.
\begin{example}
    1. Consider the sequence $1/2,1/2+1/4,1/2+1/4+1/8,\ldots$, so $x_n=1-1/2^n$ inductively.
    $\forall\epsilon>0$, we can choose $N\in\mathbb N$ such that $1/N<\epsilon$, for any $n\ge N$, $|1-x_n|=|1/2^n|\le |1/n|\le 1/N<\epsilon$, so $x_n\to 1$.\\
    2. Any constant sequence converges to that constant.
    Indeed, for any $\epsilon>0$, choose $N=1$, so for every $n>N$, the sequence is $c$ hence is $\epsilon$-close to the constant.\\
    3. Consider $x_n=(-1)^n$.
    This sequence does not converge to a limit.
    Suppose it does, then suppose the limit is $c$, then we can choose $\epsilon=1$, then there is some $N\in\mathbb N$, such that $\forall k>N, |x_n-c|<\epsilon=1$, but we can choose $n$ such that $2n>N$, but $2=|x_{2n}-x_{2n+1}|\le |x_{2n}-c|+|x_{2n+1}-c|<2\epsilon=2$, contradiction.\\
    3. Sequence needs not have a closed form, we can take
    $$x_n=
    \begin{cases}
        1/n\text{, if $n$ is odd}\\
        0\text{, otherwise}
    \end{cases}$$
    Then it is trivial that $x_n\to 0$.
    $\forall\epsilon>0$, we choose $N$ such that $N>1/\epsilon$, then $\forall n\ge N,|x_n-0|\le 1/N<\epsilon$.
\end{example}
\begin{remark}
    1. If $x_n\to c$ for some $c$, we say the sequence $(x_n)$ or $(x_n)_{n=1}^\infty$ is convergent.
    If it is not the case, then we say it is divergent.
    Note that the example $(-1)^n$ shows that a divergent sequence needs not to go to infinity.\\
    2. Limits, if exist, are unique.
    If $x_n\to c$ and $x_n\to d$, then $c=d$.
    Indeed, if $c\neq d$, then $|c-d|>0$, so we choose $\epsilon=|c-d|/2$, so $\exists N_1\in\mathbb N,\forall n\ge N, |x_n-c|<\epsilon,\exists N_2\in\mathbb N,\forall n\ge N_2, |x_n-d|<\epsilon$, so let $N=\max\{N_1,N_2\}$, so $|c-d|=2\epsilon >|x_N-c|+|x_N-d|\ge |c-d|$, contradiction.
\end{remark}
A sequence given in the form $x_1,x_1+x_2,x_1+x_2+x_3,\ldots$ is called a series.
We can write
$$\sum_{n=1}^\infty x_n$$
for the series, and the $k^{th}$ term of it is
$$\sum_{n=1}^k x_n$$
This is called a partial sum of the series.
If a series is convergent, one can also write the infinite sum as its limit, so
$$\sum_{n=1}^\infty \frac{1}{2^n}=1$$
We should not write and cannot write something like ``let $c$ be the limit of $(x_n)$ and blah blah blah'' before ensuring that the sequence does converge since such a number may not exist.\\
Limits do behave nicely.
\begin{example}
    1. If $x_n\le d$ always, and $x_n\to c$, then $c\le d$.
    Indeed, if $c>d$, then we can take $\epsilon=c-d$, so $|c-x_n|=c-x_n=(c-d)+(d-x_n)\ge c-d=\epsilon$, contradiction.\\
    2. If $x_n<d$ always, and $x_n\to c$, then we need not have $c<d$, since we can take for example $x_n=-1/n\to 0$ and $d=0$.
\end{example}
\begin{proposition}
    If $x_n\to x$, $y_n\to y$, then $z_n=x_n+y_n\to x+y$.
\end{proposition}
\begin{proof}
    $\forall\epsilon>0$, choose $N_1$ with $\forall n>N_1, |x_n-x|<\epsilon/2$ and $N_2$ with $\forall n>N_2, |y_n-y|<\epsilon/2$, then choose $N=\max\{N_1,N_2\}$, then $\forall n>N,|z_n-(x+y)|\le |x_n-x|+|y_n-y|<2\epsilon/2=\epsilon$, so $z_n\to x+y$.
\end{proof}
We say a sequence $x_1,x_2,\ldots$ is increasing if $x_n\le x_{n+1}$ for every $n\in\mathbb N$.
It is called bounded above if $\{x_n:n\in\mathbb N\}$ is bounded above.
\begin{theorem}
    An incresing sequence that is bounded above is convergent.    
\end{theorem}
Note that this is false in $\mathbb Q$.
\begin{proof}
    Suppose $(x_n)$ is such a sequence.
    We claim that it converges to $c=\sup\{x_n:n\in\mathbb N\}$.
    For any $\epsilon>0$, $c-\epsilon$ cannot be an upper bound, so there is some $N\in\mathbb N$ such that $c\ge x_N>c-\epsilon$, then for any $n>N$, $c\ge x_n\ge x_N>c-\epsilon$, so $|c-x_n|<\epsilon$.
\end{proof}
\begin{remark}
    1. It is equivalent to say that a decreasing sequence (i.e. $x_n\ge x_{n+1}$ for all $n\in\mathbb N$) bounded below is convergent.
    So as a corollary, a bounded monotone sequence is convergent.\\
    2. In series, the equivalent form would be that if the partial sums of the series is bounded above and all terms are nonnegative, then it converges by corollary just stated.
\end{remark}
There are a few applications of the theorem (and its corollaries).
\begin{proposition}
    $\sum_{n=1}^\infty n^{-1}$ diverges, and $\sum_{n=1}^\infty n^{-2}$ converges.
\end{proposition}
Note that as sequences, neither of the sums has a (nice) closed form.
Our main idea would be doing comparisons with other series whose sums we are more familiar with.
\begin{proof}
    For the first part of the proposition, choose any partial sum
    $$P=\sum_{n=1}^k n^{-1}$$
    then consider the least $r$ such that $2^{r}>k$, so
    $$\sum_{n=1}^{2^{r+1}} \frac{1}{n}\ge P+\sum_{n=2^r}^{2^{r+1}} \frac{1}{n}\ge P+\frac{2^{r}}{2^{r+1}}=P+\frac{1}{2}$$
    So it is unbounded, hence does not converge.
    For the second part,
    $$\sum_{n=2^r}^{2^{r+1}-1} \frac{1}{n^2}\le \frac{2^r}{2^{2r}}=\frac{1}{2^r}\implies \sum_{n=1}^\infty\frac{1}{n^2}\le \sum_{r=0}^\infty\frac{1}{2^r}=2$$
    So it is bounded, hence it converges.
\end{proof}
The last sum actually tends to $\pi^2/6$, the proof of this will be covered somewhere else.\\
Secondly, decimal expansion.
When we are thinking of $0.a_1a_2a_3\ldots$, how do we know the limit $0.a_1,0.a_1a_2,\ldots$ always exists?
Obviously, we want to analyze the infinite sum
$$\sum_{n=1}^\infty \frac{a_n}{10^n}$$
This does converge, since
$$\sum_{n=1}^\infty \frac{a_n}{10^n}\le \sum_{n=1}^\infty \frac{10}{10^n}<\sum_{k=0}^\infty\frac{1}{10^k}<2$$
Conversely, given any real number $0<x<1$, we first choose $a_1\in\{0,1,\ldots,9\}$ such that $a_1/10\le x<(a_1+1)/10$, and when we have chosen $a_k$, we choose $a_{k+1}\in\{0,1,\ldots,9\}$ by choosing it to be such that
$$\sum_{n=1}^k\frac{a_n}{10^n}+\frac{a_{k+1}}{10^{k+1}}\le x <\sum_{n=1}^k\frac{a_n}{10^n}+\frac{a_{k+1}+1}{10^{k+1}}$$
such $a_{k+1}$ always exists due to the definitions of $a_1,a_2,\ldots,a_k$.
\begin{remark}
    1. The decimal (or binary or other bases) expansion of any rational number is periodic.
    Conversely, a real number whose decimal (or binary or other bases) expansion is periodic is rational.\\
    2. Decimal expansion may not be unique (in a way) since $0.500\ldots =0.499\ldots$, but this is kind of the only way that this would happen.
\end{remark}
\begin{definition}
    We define $e$ to be $1+1/1!+1/2!+1/3!+1/4!+\cdots$
\end{definition}
\begin{proposition}
    $e$ is well defined.
\end{proposition}
\begin{proof}
    The partial sum is bounded since
    $$\sum_{i=0}^\infty\frac{1}{i!}\le 1+\sum_{k=0}^\infty\frac{1}{2^k}=3$$
    Also each term is positive, therefore the sequence of partial sums is increasing, hence it converges.
\end{proof}
\subsection{Transcendental Numbers}
\begin{definition}
    A real number $a$ is called algebraic if it is the root of some nonzero polynomial with integer coefficients.
\end{definition}
\begin{example}
    1. Every rational number is algebraic.\\
    2. $\sqrt[q]{p}+r$ is algebraic for any $p,q,r\in\mathbb Q$ (given that it is well-defined).
\end{example}
However, there can be real numbers that is not algebraic.
To start with, we shall show that $e$ is irrational.
\begin{proposition}
    $e$ is irrational.
\end{proposition}
\begin{proof}
    Suppose for the sake of contradiction that $e=p/q$ such that $p\in\mathbb Z,q\in\mathbb N$.
    Obviously $q>1$.
    So
    $$\frac{p}{q}=1+\frac{1}{1!}+\frac{1}{2!}+\cdots\implies p(q-1)!=q!\left(\sum_{n=0}^q\frac{1}{n!}\right)+\sum_{n=q+1}^\infty\frac{q!}{n!}$$
    Note that the first term in the right hand side is integral, but for the second term,
    $$0<\sum_{n=q+1}^\infty\frac{q!}{n!}=\frac{1}{q+1}+\frac{1}{(q+1)(q+2)}+\cdots\le\sum_{k=1}^\infty\frac{1}{(q+1)^k}=\frac{1}{q}<1$$
    Thus the right hand side is not an integer, but the left hand side is.
    This is a contradiction.
\end{proof}
\begin{definition}
    If a real number $x$ is not algebraic, we say it is transcendental.
\end{definition}
$e$ is actually transcendental, but we will not prove it here.
However, we will give an example of a transcendental number.
\begin{proposition}
    Take
    $$c=\sum_{k=1}^\infty \frac{1}{10^{k!}}$$
    then $c$ is transcendental.
\end{proposition}
It is trivial that $c$ is well defined.
We will need the following facts about polynomials.
\begin{proposition}
    1. For any polynomial $P$, we have some constant $K>0$ such that $|P(x)-P(y)|\le K|x-y|$ for each $x,y\in [0,1]$.\\
    2. A polynomial of degree $d$ has at most $d$ roots.
\end{proposition}
\begin{proof}
    1. Suppose $P(x)=\sum_{i=0}^da_ix^i$, then for $x,y\in[0,1]$,
    $$|P(x)-P(y)|=\left|\sum_{i=1}^da_i(x^i-y^i)\right|\le d\sum_{i=1}^d|a_i||x-y|=K|x-y|,K=d\sum_{i=1}^d|a_i|$$
    2. Polynomial division.
\end{proof}
Now we are ready for the proof.
\begin{proof}[Proof that $c$ is transcendental]
    Suppose for the sake of contradiction that there is some polynomial
    $$P(x)=\sum_{k=0}^da_kx^k$$
    such that $a_k\in\mathbb Z$ and $P(c)=0$.
    Let
    $$c_n=\sum_{k=1}^n \frac{1}{10^{k!}}$$
    So $c_n\to c$.
    Note that $|c_n-c|\le 2/10^{(n+1)!}$.
    Since $P$ has at most $d$ roots, $\exists N\in\mathbb N,\forall n>N,P(c_n)\neq 0$.
    As $P$ has integer coefficients, for $n>N$,
    $$\frac{1}{10^{n!d}}\le|P(c_n)|=|P(c_n)-P(c)|\le K|c_n-c|\le\frac{2K}{10^{(n+1)!}}$$
    by the preceding proposition.
    This fails when $n$ is large, contradiction.
\end{proof}
\begin{definition}
    We say $x\in\mathbb R\setminus\mathbb Q$ is a Liouville number if $\forall n\in\mathbb N,\exists p\in\mathbb Z,q\in\mathbb N,$
    $$\left|x-\frac{p}{q}\right|<\frac{1}{q^n}$$
    That is, $x$ has a very good rational approximations.
\end{definition}
\begin{theorem}
    Every Liouville number is transcendental.
\end{theorem}
\begin{proof}
    Similar as above.
\end{proof}
\begin{remark}
    $e$ is not a Liouville number.
\end{remark}
