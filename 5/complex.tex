\section{The Complex Numbers}
Some polynomials have no solution in $\mathbb R$, e.g. $x^2+1=0$.
So we want so enlarge the reals to seek a solution.
\begin{definition}
    The complex numbers, written as $\mathbb C$, consists of $\mathbb R^2$ (ordered pairs of real numbers) with operations $+,\cdot$ defined by
    $$(a,b)+(c,d)=(a+c,b+d)$$
    $$(a,b)\cdot(c,d)=(ac-bd,ad+bc)$$
\end{definition}
The collection $\{(a,0):a\in\mathbb R\}$ can be identified as the reals.
One can verify straight away that the sum and product reduced to the case that we are familiar with.
We can write $i=(0,1)$, and we have $i^2=)=(-1,0)$, also every complex numbers are in the form $a+bi$ for $a,bin\mathbb R$.
Indeed, $(a,b)=(a,0)+(b,0)\cdot (0,1)=a+bi$.
\begin{remark}
    We can check that the complex numbers satisfy the usual laws of arithmetic.
    In particular, for each $z=a+bi\neq 0$, we know that $zw=1$ where $w=\bar z/(z\bar z)$ where $\bar z=a-bi$.
    The complex number $\bar z$ is called the conjugate of $z$.
    Such a structure is called a field.
\end{remark}
Examples of fields includes $\mathbb C,\mathbb R,\mathbb Q,\mathbb Z_p$ where $p$ is prime.
But $\mathbb Z$ is not due to the lack of multiplicative inverses.
\begin{theorem}[Fundamental Theorem of Algebra]
    Every nonconstant polynomial with complex coefficient has a root in $\mathbb C$.
\end{theorem}
We will obviously not prove it here.