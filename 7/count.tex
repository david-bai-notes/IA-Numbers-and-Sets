\section{Countability}
We want to find a way to describe the sizes of infinite sets.
For example, $\mathbb N$ ``looks smaller'' than $\mathbb Z,\mathbb Q, \mathbb R$, but is that the case?
\begin{definition}
    A set $A$ is called countable if either $A$ is finite or there exists a bijection $A\to\mathbb N$.
\end{definition}
Equivalently, $A$ is countable iff we can list the element of $A$ as $\{a_1,a_2,a_3,\ldots\}$ (which might terminate if $A$ is finite).
\begin{example}
    1. Every finite set is countable.\\
    2. $\mathbb N$ is countable.\\
    3. $\mathbb Z$ is countable.
    Consider the listing $\{0,1,-1,2,-2,3,-3,\ldots\}$.
    Or, written in formula
    $$a_n=
    \begin{cases}
        n/2\text{, if $n$ is even}\\
        (1-n)/2\text{, if $n$ is odd}
    \end{cases}$$
\end{example}
A natural question to ask, following the last example, is whether all sets are countable.
For example, is $\mathbb Q$ countable?
How about $\mathbb R$?
\begin{proposition}
    Let $A$ be a set, then $A$ is countable iff there is an injection $f:A\to\mathbb N$
\end{proposition}
\begin{proof}
    If $A$ is countable, it is obvious that there is an injection.\\
    Conversely, it is done if $A$ is finite, so we can henceforth assume that $A$ is infinite.
    Note that $f(A)=\operatorname{Im}f$ is a subset of $\mathbb N$ and $f$ is a bijection $A\to f(A)$.
    So it is enough to show that $f(A)$ is countable.
    We can order $f(A)$ like we did in $\mathbb N$, then define $b_n$ recursively by $b_1=\min f(A)$ and $b_{n+1}=\min (f(A)\setminus\{b_i:1\le i\le n\})$.
    So the sequence $\{b_n\}_{n\in\mathbb N}$ lists $f(A)$, hence $f(A)$ is countable, so $A$ is countable.
\end{proof}
\begin{corollary}
    Every subset of a countable is countable.
\end{corollary}
\begin{remark}
    Consider the set $\{n/(n+1):n\in\mathbb N\}\cup\{1\}\subset \mathbb R$, then this set is countable, but we cannot hit everything by writing every element in increasing order.
\end{remark}
\begin{theorem}
    $\mathbb N\times\mathbb N$ is countable.
\end{theorem}
\begin{proof}
    Define the listing $a_1,a_2,\ldots\in\mathbb N\times\mathbb N$ by $a_1=(1,1)$ and if we have $a_n=(p,q)$, then
    $$a_{n+1}=\begin{cases}
        (p-1,q+1)\text{, if $p>1$}\\
        (p+q,1)\text{, otherwise}
    \end{cases}$$
    This lists all points in $\mathbb N\times\mathbb N$.
    All $(x,y)\in\mathbb N\times\mathbb N$ are hit by induction on $x+y$.
\end{proof}
\begin{proof}[Alternative proof]
    Consider the map $f:\mathbb N\times\mathbb N\to\mathbb N$ by $(a,b)\mapsto 2^a3^b$ is an injection.
\end{proof}
More generally, the same shows
\begin{corollary}\label{union_countable}
    Let $\{A_i\}_{i\in\mathbb N}$ be a collection of countable sets, then $\bigcup_{i\in \mathbb N}A_i$ is countable.
\end{corollary}
\begin{proof}
    Each $A_i$ is countable, so we can list $A_i$ as $\{a_{i1},a_{i2},\ldots\}$ (which might terminate if $A_i$ is finite).
    Now consider the function $f:\bigcup_{i\in \mathbb N}A_i\to\mathbb N$ by $x\mapsto 2^i3^j$ where $x=a_{ij}$ such that $i$ is the least such that $x\in A_i$ and $j$ is the least such that $x=a_{ij}$.
    This is an injection.
\end{proof}
\begin{example}
    1. $\mathbb Q$ is countable.
    Indeed,
    $$\mathbb Q=\bigcup_{n\in\mathbb N}\frac{1}{n}\mathbb Z$$
    So we are done by the preceding corollary.
    Alternatively, there is an obvious injection from $\mathbb Q$ to $\mathbb Z\times\mathbb Z$.\\
    2. The set $\mathbb A$ of all algebraic numbers is countable.
    Indeed, each polynomial has only finitely many roots, so it is enough to show that there are couontably many integer polynomials, then the claim is proved by Corollary \ref{union_countable}.
    But again by this corollary it is enough to show that there are only countably many integer polynomials of degree $d$ for any $d\in\mathbb N$.
    However this set injects to $\mathbb Z^{d+1}$ by $a_0+a_1x+\cdots+a_dx^d\mapsto (a_0,a_1,\ldots,a_d)$, so it is countable.
\end{example}
We have got many many countable sets in our stock, so the question is then whether all sets are countable.
\begin{theorem}
    $\mathbb R$ is uncountable (that is, not countable).
\end{theorem}
\begin{proof}
    It suffices to show that $(0,1)$ is uncountable.
    So given any sequence $r_1,r_2,\ldots$ of $(0,1)$, we shall show that there is some $s\in (0,1)$ that is not of the form $r_i$.
    We write each number in $\{r_i:i\in\mathbb N\}$ as decimals.
    That is, $r_i=0.r_{i1}r_{i2}\ldots$.
    For each $n\in\mathbb N$, choose a digit $s_n\in\{5,6\}\setminus\{r_{nn}\}\neq\varnothing$,
    \footnote{There is nothing special about $5,6$, we just need to get rid of the case of $9999\ldots$}
    then $0.s_1s_2\ldots\in (0,1)$ is not of the form $a_i$.
\end{proof}
The above proof is called Cantor's Diagonal Argument.
\begin{remark}
    $\mathbb A\cap \mathbb R$ is countable but $\mathbb R$ is not, so there exists transcendental numbers.
    In fact, ``most'' reals are transcendental, as the set $\mathbb R\setminus\mathbb A$ is uncountable.
\end{remark}
\begin{theorem}
    $2^{\mathbb N}$ is uncountable.
\end{theorem}
\begin{proof}
    Suppose $2^{\mathbb N}=\{s_1,s_2,\ldots\}$.
    Consider the set $S\subset\mathbb N$ by $n\in S\iff n\notin s_n$, that is $\{n\in\mathbb N:n\notin s_n\}$, so $S\neq s_i$ for any $i$, contradiction.
\end{proof}
\begin{remark}
    This really is exactly the proof idea of $\mathbb R$ being uncountable.
    Alternatively, we can actually inject $(0,1)$ to $2^\mathbb N$ by writing in base $2$.
\end{remark}
The proof above can be actually extended to the following:
\begin{theorem}
    Let $X$ be a set, then there is no bijection $X\to 2^X$.
\end{theorem}
For example, there is no bijection between $\mathbb R$ and $2^{\mathbb R}$.
\begin{proof}
    Let $f:X\to 2^X$ be a function.
    Consider the set $S=\{x\in X:x\notin f(x)\}$.
    But $S\notin f(X)$ since for any $x\in X,S\neq f(x)$ (as $x\in S\iff x\notin f(x)$), so $f$ is not surjective.
\end{proof}
Note that the set $S=\{x\in X:x\notin f(x)\}$ is very similar to Russel's Paradox.\\
Now we want to explore more countability arguments.
\begin{example}
    Any collection $\{A_i\}_{i\in I}$ of pairwise disjoint intervals is countable.
    The first proof that we can do is to inject the set into the rational numbers by choosing one from each interval.
    The second proof is by observing that the numbers of intervals in the family with length at least $1/n$ is countable, then we can write our family as a countable union of countable sets, hence is countable.
\end{example}
There are generally two sorts of arguments to show a set is uncountable, either copying the diagonal argument or inject an uncountable set to it.
To show it is countable, either we can list it (which however often get fiddly) or we can inject it into a countable set.
Another way is to use the fact that a countable union of countable sets is countable.\\
Now consider $A,B$ nonempty.
Intuitively, we think of $A$ bijecting with $B$ as saying that the size of $A$ is the size of $B$.
Similarly, we think of $A$ injecting into $B$ as $A$ has at most as large as $B$, and $A$ surjecting to $B$ as $A$ has at least as large as $B$.
For these to make sense, we certainly want that $A$ injects into $B$ if and only if $B$ surjects into $A$.
Indeed, given $f:A\to B$ injective, then consider $a\in A$, then $g:B\to A$ by
$$b\mapsto\begin{cases}
    f^{-1}(b)\text{, if $b\in f(A)$}\\
    a\text{, otherwise}
\end{cases}$$
Then $g$ is surjective.
Conversely, given $g:B\to A$ surjective, consider $f:A\to B$ with $a\mapsto a'$ by choosing an $a'\in g^{-1}(\{a\})$.
\footnote{Does it depend on the Axiom of Choice?}
Also it is intuitive that if $A$ injects into $B$ and $B$ injects into $A$, then $A$ bijects with $B$.
\begin{theorem}[Schr\"oder–Bernstein]
    Let $f:A\to B$ and $g:B\to A$ be injective, then there is a bijection $h:A\to B$
\end{theorem}
\begin{proof}
    For $a\in A$, we write $g^{-1}(a)$ for the unique (if exists) point in $B$ such that $g(g^{-1}(a))=a$.
    Define similarly $f^{-1}(b)$ for $b\in B$.
    The ``ancestors'' of $a\in A$ consists of
    $$g^{-1}(a), f^{-1}(g^{-1}(a)),g^{-1}(f^{-1}(g^{-1}(a))),f^{-1}(g^{-1}(f^{-1}(g^{-1}(a)))),\ldots$$
    which may or may not terminate.
    Similarly for $b\in B$.
    Let $A_0$ be the set of $a\in A$ such that the ancestor sequence of $a$ that terminates in even time, that is it has even length (so it includes those $a\notin g(B)$ since $0$ is even).
    And $A_1$ be the set of $a\in A$ such that the ancestor sequence of $a$ that terminates in odd time, $A_\infty$ be those whose ancestor sequence does not terminate.
    Similarly construct $B_0,B_1,B_\infty$.
    So $A_0,A_1,A_\infty$ partitions $A$ and $B_0,B_1,B_\infty$ partitions $B$.
    By definition, $f|_{A_0}$ is a bijection $A_0\to B_1$ (as every $b\in B_1$ is in the image of $f|_{A_0}$).
    And simiarly, $g|_{B_0}$ is a bijection $B_0\to A_1$.
    As for infinity cases, $f|_{A_\infty}$ is a bijection $A_{\infty}\to B_\infty$, so the function $h$ can be defined by
    $$h(a)=
    \begin{cases}
        f(a)\text{, if $a\notin A_1$}\\
        g^{-1}(a)\text{, if $a\in A_1$}
    \end{cases}$$
    Then $h:A\to B$ is well-defined and bijective.
\end{proof}
\begin{example}
    $[0,1]$ and $[0,1]\cup [2,3]$ biject.
    Indeed, $x\mapsto x$ is an injection from $[0,1]\to [0,1]\cup [2,3]$.
    Also $x\mapsto x/3$ injects $[0,1]\cup[2,3]$ to $[0,1]$
\end{example}
Now, is it true that for any sets $A,B$, either $A$ injects into $B$ or vice versa?
The answer is yes, but very hard and way beyond the scope of this course.\\
Now given $\mathbb N$, we can construct a strictly increasing sequence (in terms of sizes) $\mathbb N,2^{\mathbb N}, 2^{2^{\mathbb N}},\ldots$, but does every set injects into one of them?
The answer is obviously no, since we can take $X=\mathbb N\cup 2^{\mathbb N}\cup 2^{2^{\mathbb N}}\cup\cdots$.
Now this is definitely not the biggest set either, since we can now take again $X,2^X, 2^{2^X},\ldots$, which is again beaten by the union $X'$ of all of them.
We can then construct a sequence $X,X',X'',\ldots$, which is again beaten by $X\cup X'\cup X''\cup\cdots$.
And we can do the same thing again and again and again and this never ends... (but the course does here).